% The \phantomsection command is needed to create a link to a place in the document that is not a
% figure, equation, table, section, subsection, chapter, etc.
%
% When do I need to invoke \phantomsection?
% https://tex.stackexchange.com/questions/44088/when-do-i-need-to-invoke-phantomsection
\phantomsection

% ---
% Considerações Finais (outro exemplo de capítulo sem numeração e presente no sumário)
% ---

\chapter[Conclusion]{\lang{Conclusion}{Considerações Finais}} \label{sec:conclusions}
In this work we proposed SMSM, a new similarity measure for semantic trajectories that supports both stops and moves.  To the best of our knowledge, SMSM is the first semantic trajectory similarity measure that deals with both stops and moves and their space, time and semantic dimensions. The proposed similarity measure is robust  to consider multiple dimensions of stops and moves, where a move, for instance, can be represented as raw points, the traveled distance, the major direction, the names of streets, the transportation mode, etc.

SMSM supports the definition of weights for stops, moves and dimensions, so the measure is flexible to give more or less importance for specific parts of trajectories. On the other hand, these weights may be difficult to estimate from the user point of view, but in case he has no knowledge about the domain, the best is to define the same weight for all elements.

We performed experiments using real and synthetic data of distinct contexts, including car trajectories and pedestrian trajectories. By evaluating SMSM with an information retrieval approach, we show that SMSM was more accurate than other measures developed either for raw or semantic trajectories.

SMSM requires a full spatial match between the start and end stop of two movement elements to evaluate the move. In future works we will study an extension of SMSM to evaluate the move in cases where the final stops of two movement elements do not have a match.
