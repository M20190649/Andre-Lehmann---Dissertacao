\phantomsection


% Is it possible to keep my translation together with original text?
% https://tex.stackexchange.com/questions/5076/is-it-possible-to-keep-my-translation-together-with-original-text
\chapter[\lang{Related works}{Trabalhos relacionados}]
{
    \lang
    {Related works}
    {Trabalhos relacionados}
}
\label{sec:related}

\begin{flushright}
    \englishword{ }
\end{flushright}


% Multiple-language document - babel - selectlanguage vs begin/end{otherlanguage}
% https://tex.stackexchange.com/questions/36526/multiple-language-document-babel-selectlanguage-vs-begin-endotherlanguage

Along the last years, many similarity measures for trajectories were proposed, focusing on raw trajectories, which find the similarity between two trajectories considering spatial and temporal information. Vlachos in \cite{vlachos2002discovering} proposed LCSS, a measure in which two points \textit{match} when their distance is less than a given threshold. The longer the common subsequence of matches between two trajectories, the more similar they are. {Edit Distance with Real Penalty (ERP) is a distance function proposed for time series in }\cite{Chen:2004:MLE:1316689.1316758}. It computes the distance between two sequences of points by aligning the sequences, allowing gaps in the sequence when there are points that do not match. The gaps are penalized based on the distance of the unmatched points. Chen in \cite{Chen:2005:RFS:1066157.1066213} proposed EDR, a similarity measure for raw trajectories that calculates the edition difference between the points, as classical edit distance measures. Another measure for raw trajectories is {w-constrained discrete Frechet distance} (wDF), proposed in \cite{Ding:2008:ESJ:1440463.1440989}, considering only the spatial dimension. Although LCSS and EDR have not been proposed for this intent, both measures can be easily extended to handle other dimensions (e.g. semantics). However, that extensibility does not allow the measures to represent semantic trajectories as a sequence of heterogeneous elements, that is the case of stops and moves, because both LCSS and EDR demand that all trajectory elements should be homogeneous.

The distance measure Dynamic Time-Warping adaptive (DTWa) proposed in \cite{Shokoohi-Yekta2017} extends the classical {Dynamic Time-Warping} (DTW) \cite{berndt1994using} distance measure to multiple data dimensions. The problem of DTWa is that it deals with numeric dimensions only, and it is a distance function, not a similarity measure.

{UMS was proposed in \cite{Furtado-UMS-2018} as a new parameter-free similarity measure for raw trajectories. UMS was designed exclusively for raw trajectories, considering only the spatial dimension. The major contribution of UMS is that no distance threshold is needed, and the distance between two trajectories is computed with ellipses over every two trajectory points, so trajectories are represented as sequences of ellipses. The similarity is given by the proportion of ellipse intersection. As the ellipses are dynamically defined according to the point distance, UMS solves the problem of trajectories with distinct or varied sampling rates, by computing each ellipse size as big as necessary to cover two consecutive points. UMS outperformed all state-of-the-art works for raw trajectories developed before 2018, so it is currently the best similarity measure for raw trajectories where the spatial dimension is the most important.}

%\hl{Distinctly of the raw points measures, in the work of }\cite{Laube2005}\hl{ a raw trajectory is analyzed through the derived features of the movement, such as speed changing, momentous speed and others. The main purpose of work of Laube is to define a few behavior patterns in raw trajectories and find spatial encounters in respect to these behaviors. Using the REMO concept (RElative MOtion), it approach has as constraint that analyzed trajectories must be synchronized in time. This is not the case in real life trajectories of persons or regular vehicles, since each moving object (person, vehicles, etc) starts its movement independently each other, limiting it uses for some specific contexts, such as birds migration, wildlife monitoring, etc. In a similar way, the work proposed in } \cite{Shirabe2006}\hl{ shows that a strong correlation of motion features can help to find a linear correlation between trajectories in raw trajectory datasets. A drawback of the approach is the inability to consider the spatial and temporal location of each point. In other words, the trajectories are analyzed only by their behavior and not by their discrete points.}

The Common Visit Time Interval (CVTI) is proposed in \cite{Kang:2009:SMT:1529282.1529580} as a measure to integrate the semantic dimension of the stops with the temporal dimension. Although using different data dimensions, the measure is not extensible for other dimensions associated with the point, not allowing heterogeneous data such as stops and moves to be modeled and measured together.

In \cite{Ying:2010:MUS:1867699.1867703} the measure {Maximal Semantic Trajectory Pattern} (MSTP) is proposed, which despite being a measure for semantic trajectories, it is not able to handle multiple data dimensions. Moreover, as MSTP essentially works with the frequency at which stops are visited, it is not able to consider moves between stops.

In the work of \cite{Furtado:TGIS12156}, the MSM measure is proposed to multidimensional similarity measuring, and it is so far, the best similarity measure for semantic trajectories. In MSM all elements must be homogeneous, not allowing the representation of heterogeneous elements as stops and moves. MSM is more robust than LCSS and EDR by allowing partial dimension matching and not forcing a sequence. MSM was developed to work only with stops, and the sequence of elements is not taken into account during the similarity calculation, ignoring the order of the stops. We claim that for some applications as car sharing, new bus route planning, traffic analysis, and recommendation systems, the sequence of stops may play an important role.
