\swapcontents
{
    \addtotextpreliminarycontent{Resumo em Português}
    \begin{otherlanguage*}{brazil}
    \begin{resumo}[Resumo]

        Por muitos anos a pesquisa em similaridade de trajetórias focou nas trajetórias brutas, considerando somente a informação de espaço e tempo. {Medidas de similaridade são a base para a maioria dos métodos de mineração de dados e extração de conhecimento. }Com o enriquecimento das trajetórias com informações semânticas, como o nome e a categoria dos locais visitados, meio de transporte utilizado, nome das ruas percorridas, etc, emergiu a necessidade por medidas de similaridade que suportem espaço, tempo e semântica. Apesar de algumas medidas de similaridade para trajetórias lidarem com todas estas dimensões, elas consideram somente os locais onde o objeto móvel faz paradas, denominados \emph{stops}, ignorando o movimento que ocorre entre as paradas, denominados \emph{moves}.
        Acredita-se que, para algumas aplicações, o movimento entre os \emph{stops} é tão importante quanto o \emph{stop} em si, e ele deve ser levado em consideração na análise da similaridade, como em sistemas de transporte público, planejamento urbano, entre outros.
        Nesta dissertação é proposta a medida Similarity Measure for trajectory Stops and Moves (SMSM), um nova medida de similaridade para trajetórias semânticas que considera tanto os \emph{stops} quanto os \emph{moves}.
        O SMSM é avaliado em três conjuntos de dados: (i) um conjunto de dados de trajetórias sintéticas criadas com o gerador de trajetórias semânticas Hermoupolis, (ii) um conjunto de trajetórias reais de táxis do projeto CRAWDAD, e (iii) o conjunto de dados de trajetórias reais chamado Geolife, com trajetórias de pessoas na cidade de Pequim. Os resultados mostram que o SMSM supera as medidas de similaridade do estado da arte desenvolvidas tanto para trajetórias brutas quanto semânticas.
        
        \imprimirpalavraschave{Palavras-chaves}{\begin{inparaitem}[]\palavraschaveportugues\end{inparaitem}}

    \end{resumo}
    \end{otherlanguage*}
}
{
    \addtotextpreliminarycontent{Resumo expandido em Português}
    \begin{otherlanguage*}{brazil}
    \begin{resumo}[Resumo Expandido]
        \textbf{INTRODUÇÃO}
        \newline
        \newline
        {Calcular a similaridade de duas trajetórias é importante para que perguntas como "Dentre um conjunto \emph{M} de trajetórias existentes, quais são as mais parecidas com uma trajetória \emph{s}?" ou "Quais são os pares de trajetórias mais semelhantes em um conjunto \emph{M} de trajetórias?". Para tanto, é importante a confecção de medidas de similaridade para trajetórias.} Por muitos anos a pesquisa em similaridade de trajetórias focou nas trajetórias brutas, que são sequências de pontos com informações de localização e tempo. Estas medidas de similaridade somente consideravam a informação espaço-temporal, limitando a comparação das trajetórias às suas características geo-espaciais. Com o advento das redes sociais e o enriquecimento das trajetórias com informações semânticas, como o nome e a categoria dos locais visitados, meio de transporte utilizado no deslocamento, o nome das ruas percorridas, etc, emergiu a necessidade por medidas de similaridade que suportem as \emph{trajetórias semânticas}, onde cada ponto possui espaço, tempo e semântica. Apesar de algumas medidas de similaridade para trajetórias lidarem com este novo tipo de trajetória, elas consideram somente os locais visitados pelo objeto móvel, denominados \emph{stops}, ignorando aquilo que ocorre entre os locais, denominados \emph{moves}.
        Acredita-se que para algumas aplicações, o movimento entre os \emph{stops} é tão importante quanto o \emph{stop} em si, e ele deve ser levado em consideração na análise de similaridade. {Por exemplo em sistemas de gerenciamento de tráfego, sistemas de transporte público, planejamento urbano, entre outros.}
        Nesta dissertação é proposta a medida SMSM (Similarity Measure for trajectory Stops and Moves), uma nova medida de similaridade para trajetórias semânticas que considera tanto os \emph{stops} quanto os \emph{moves} nas trajetórias semânticas.
        O SMSM é avaliado em três conjuntos de dados: (i) um conjunto de dados de trajetórias sintéticas criadas com o gerador de trajetórias semânticas Hermoupolis, (ii) um conjunto de trajetórias reais {de táxis} do projeto CRAWDAD, e (iii) o conjunto de dados de trajetórias reais chamado Geolife, com trajetórias de pessoas na cidade de Pequim. Os resultados mostram que o SMSM supera as medidas de similaridade do estado da arte desenvolvidas tanto para trajetórias brutas quanto semânticas.
        
        \newline
        \newline
        \textbf{OBJETIVOS}
        \newline
        \newline
        O objetivo geral deste trabalho é a proposição de uma nova medida de similaridade para trajetórias semânticas.
        \newline
        {O objetivo específico desta dissertação é propor uma nova medida de similaridade para trajetórias semânticas que considera tanto os \emph{stops} quanto os \emph{moves} das trajetórias. A nova medida provê suporte de múltiplas dimensões como espaço, tempo, semântica e quaisquer outras dimensões adicionais, atribuindo diferentes pesos para cada dimensão conforme a aplicação, além de considerar parcialmente a ordem dos pontos na trajetória semântica.}
        \newline
        \newline
        
        \textbf{METODOLOGIA}
        \newline
        \newline
        Inicialmente foi realizada uma revisão da literatura em tópicos relacionados à similaridade de trajetórias semânticas, através de ferramentas de pesquisa como Google Scholar e em periódicos e conferências de alto impacto (como TKDE, IJGIS, TGIS, VLDB, DKE, ACMSIGSpatial, entre outros). Através da análise e implementação das medidas existentes foram identificadas algumas das suas limitações e com isto foi possível propor uma nova medida de similaridade que melhore a medição de similaridade entre duas trajetórias semânticas {e que acima de tudo conseguisse tratar todas as partes das trajetórias e suas dimensões.}
        \newline
        A medida SMSM considera que o \emph{move} nas trajetórias semânticas é relevante e deveria ser considerado na medição de similaridade de duas trajetórias. O SMSM é proposto como uma medida de similaridade para trajetórias semânticas que permite {também definir graus de importância as diferentes partes da trajetória como:} (i) o grau de importância entre os \emph{stops} e os \emph{moves}, (ii) o grau de importância de cada atributo que compõe os \emph{stops} e os \emph{moves} e também os limiares (\emph{thresholds}) utilizados em cada atributo para definir se houve casamento (\emph{matching}) ou não entre os pontos.
        \newline
        A medida foi avaliada em conjuntos de trajetórias reais, já utilizadas na literatura como \cite{epfl-mobility-20090224} e também \cite{zheng2009mining}, assim como um conjunto de dados sintéticos, gerados com a ferramenta Hermoupolis \cite{Pelekis-Hermoupolis}. 
        Inicialmente as bases de dados de trajetórias brutas foram enriquecidas com informações sobre os \emph{stops} e os \emph{moves} que ocorreram durante cada trajetória. %Este enriquecimento adiciona às trajetórias informações sobre os locais (POIs) em que cada \emph{stop} ocorreu, assim como informações sobre a atividade e o meio de transporte que foram utilizados durante cada \emph{move}.
        Com as trajetórias semanticamente enriquecidas, foi possível avaliar e comparar a medida proposta. Para isto foi utilizada a abordagem de precisão em diferentes níveis de cobertura \cite{BaezaYatesRibeiroNeto2011} em tarefas de recuperação da informação. Também foram avaliados o impacto dos parâmetros de grau de importância e limiares para a medida proposta e o tempo de execução da tarefa de recuperação de informação.
        \newline
        \newline
        
        \textbf{RESULTADOS E DISCUSSÃO}
        \newline
        \newline
        Os resultados obtidos evidenciam que a medida SMSM mostrou-se a mais robusta para avaliar a similaridade de trajetórias semânticas onde tanto as informações sobre os \emph{stops} quanto os \emph{moves} são relevantes. A medida também mostrou-se flexível para suportar múltiplas dimensões de dados tanto nos \emph{stops} quanto nos \emph{moves} e flexível também ao permitir a definição de diferentes limiares (\emph{thresholds}) para cada dimensão e graus de importância.
        
        \newline
        \newline
        \textbf{CONSIDERAÇÕES FINAIS}
        \newline
        \newline
        As principais contribuições desta dissertação são uma medida de similaridade para trajetórias semânticas que considera tanto os \emph{stops} quantos os \emph{moves}, suportando dimensões espaciais, temporais e semânticas, permitindo o uso de diferentes funções de distância para cada dimensão. A medida de similaridade é flexível suficiente para considerar parcialmente a ordem dos \emph{stops}, e suporta diferentes pesos para os \emph{stops}, os \emph{moves} e as dimensões constituintes de cada elemento, possibilitando atribuir maior ou menor importância para cada elemento. A medida proposta nesta dissertação foi publicada no periódico International Journal of Geographical Information Science.
        \imprimirpalavraschave{Palavras-chaves}{\begin{inparaitem}[]\palavraschaveportugues\end{inparaitem}}

    \end{resumo}
    \end{otherlanguage*}
}
{
    % Changing babel package inside a single chapter
    % https://tex.stackexchange.com/questions/20987/changing-babel-package-inside-a-single-chapter
    %
    % Multiple-language document - babel - selectlanguage vs begin/end{otherlanguage}
    % https://tex.stackexchange.com/questions/36526/multiple-language-document-babel-selectlanguage-vs-begin-endotherlanguage
    \addtotextpreliminarycontent{English's Abstract}
    \begin{otherlanguage*}{english}
    \begin{resumo}[Abstract]

        For many years trajectory similarity research has focused on raw trajectories, considering only space and time information. With the trajectory semantic enrichment, using information as the name and type of the visited places, the transportation mean, the name of the streets, etc,  emerged the need for similarity measures that support space, time, and semantics. Although some trajectory similarity measures deal with all these dimensions, they consider only the places where the moving object stays for constant a time, called \emph{stop}, ignoring {where the moving object pass over between two places, called \emph{move}}. We claim that, for some applications, the movement between stops is as important as the stops, and it must be considered in the similarity analysis.
        In this thesis we propose the similarity measure called {Similarity Measure for trajectory Stops and Moves(SMSM)}, a novel similarity measure for semantic trajectories that considers both stops and moves.
        We evaluate SMSM with three trajectory datasets: (i) a synthetic trajectory dataset generated with the Hermoupolis semantic trajectory generator, (ii) a real trajectory dataset {of taxis} from the CRAWDAD project, and (iii) {the Geolife trajectory dataset, with raw trajectories of persons around Beijing}. The results show that SMSM overcomes state-of-the-art measures developed for both raw and semantic trajectories.

        \imprimirpalavraschave{Keywords}{\begin{inparaitem}[]\palavraschaveingles\end{inparaitem}}

    \end{resumo}
    \end{otherlanguage*}
}