\swapcontents
{
    \addtotextpreliminarycontent{Resumo em Português}
    \begin{otherlanguage*}{brazil}
    \begin{resumo}[Resumo]

        Por muitos anos a pesquisa em similaridade de trajetórias focou nas trajetórias brutas, considerando somente a informação de espaço e tempo. Com o enriquecimento \textcolor{blue}{das trajetórias com informações semânticas, como o nome e a categria dos locais visitados, meio de transporte utilizado, nome das ruas trafegadas, etc}, emergiu a necessidade por medidas de similaridade que suportem espaço, tempo e semântica. Apesar de algumas medidades de similaridade para trajetórias lidarem com todas estas dimensões, elas consideram somente os \textcolor{blue}{locais onde o objeto móvel faz paradas (os \emph{stops}), ignorando aquilo que ocorre entre as paradas (os \emph{moves})}.
        %\textcolor{red}{de onde vieram stops e moves? precisa explicar de onde vieram, relacionar com a semantica, e falar o que sao stops e o que sao moves, porque no paragrafo seguinte voce motiva a necessidade de usar moves na similaridade, entao o leitor precisa saber o que sao}
        Acredita-se que, para algumas aplicações, o movimento entre os \emph{stops} é tão importante quanto o \emph{stop} em si, e ele deve ser levado em consideração na análise da similaridade\textcolor{blue}{, como em sistemas de gerenciamento de tráfego, transporte público, planejamento urbano, entre outros.}
        Nesta dissertação é proposta a medida \textcolor{blue}{Similarity Measure for trajectory Stops and Moves} (SMSM), um nova medidade de similaridade para trajetórias semânticas que considera tanto os \emph{stops} quanto os \emph{moves}.
        O SMSM é avaliado em três conjuntos de dados: (i) um conjunto de dados de trajetórias sintéticas criadas com o gerador de trajetórias semânticas Hermoupolis, (ii) um conjunto de trajetórias reais \textcolor{blue}{de táxis} do projeto CRAWDAD, e (iii) um conjunto de dados Geolife\textcolor{blue}{, com trajetórias reais de pessoas na cidade de Pequim}. Os resultados mostram que o SMSM supera as medidas de similaridade estado-da-arte desenvolvidas tanto para trajetórias brutas quanto semânticas.
        
        \imprimirpalavraschave{Palavras-chaves}{\begin{inparaitem}[]\palavraschaveportugues\end{inparaitem}}

    \end{resumo}
    \end{otherlanguage*}
}
{
    \addtotextpreliminarycontent{Resumo expandido em Português}
    \begin{otherlanguage*}{brazil}
    \begin{resumo}[Resumo Expandido]
        \textbf{INTRODUÇÃO}
        \newline
        \newline
        Por muitos anos a pesquisa em similaridade de trajetórias focou nas trajetórias brutas, \textcolor{blue}{que são sequências de pontos possuindo somente informações de localização e um instante de tempo. Estas medidas de similaridade somente consideravam a informação espaço-temporal, limitando a comparação das trajetórias às suas características geo-espaciais. Com o advento das redes sociais e} o enriquecimento \textcolor{blue}{das trajetórias com informações semânticas, como o nome e a categria dos locais visitados, meio de transporte utilizado, nome das ruas trafegadas, etc}, emergiu a necessidade por medidas de similaridade que suportem espaço, tempo e semântica. Apesar de algumas medidades de similaridade para trajetórias lidarem com todas estas dimensões, elas consideram somente os \textcolor{blue}{locais onde o objeto móvel faz paradas (os \emph{stops}), ignorando aquilo que ocorre entre as paradas (os \emph{moves})}.
        Acredita-se que, para algumas aplicações, o movimento entre os \emph{stops} é tão importante quanto o \emph{stop} em si, e ele deve ser levado em consideração na análise da similaridade\textcolor{blue}{, como em sistemas de gerenciamento de tráfego, transporte público, planejamento urbano, entre outros.}
        Nesta dissertação é proposta a medida \textcolor{blue}{Similarity Measure for trajectory Stops and Moves} (SMSM), um nova medidade de similaridade para trajetórias semânticas que considera tanto os \emph{stops} quanto os \emph{moves}.
        O SMSM é avaliado em três conjuntos de dados: (i) um conjunto de dados de trajetórias sintéticas criadas com o gerador de trajetórias semânticas Hermoupolis, (ii) um conjunto de trajetórias reais \textcolor{blue}{de táxis} do projeto CRAWDAD, e (iii) um conjunto de dados Geolife\textcolor{blue}{, com trajetórias reais de pessoas na cidade de Pequim}. Os resultados mostram que o SMSM supera as medidas de similaridade estado-da-arte desenvolvidas tanto para trajetórias brutas quanto semânticas.
        
        \newline
        \newline
        \textbf{OBJETIVOS}
        \newline
        \newline
        O objetivo geral deste trabalho é a proposição de uma nova medida de similaridade para trajetórias semânticas.
        \newline
        \textbf{Os objetivos específicos desta dissertação são:}
        \newline
        $\bullet$ A nova medida de similaridade para trajetórias semânticas irá considerar tanto os \emph{stops} quanto os \emph{moves} das trajetórias;
        \newline
        $\bullet$ A nova medida de similaridade permitirá medir a similaridade de trajetórias semânticas considerando múltiplas dimensões como espaço, tempo, semântica e quaisquer outras dimensões adicionais;
        \newline
        $\bullet$ A nova medida de similaridade irá considerar a ordem dos pontos na trajetórias.
        \newline
        \newline
        
        \textbf{METODOLOGIA}
        \newline
        \newline
        Inicialmente foi realizada uma revisão da literatura em tópicos relacionados à similaridade de trajetórias semânticas, através de ferramentas de pesquisa como Google Scholar e em periódicos e conferências de alto impacto (como TKDE, IJGIS, TGIS, VLDB, DKE, ACMSIGSpatial, entre outros). Através da análise e implementação das medidas existentes foram identificadas algumas das suas limitações e com isto pode-se propor uma nova medida de similaridade que melhore a medição de similaridade entre duas trajetórias.
        \newline
        A medida SMSM considera que o \emph{move} nas trajetórias semânticas é relevante e deveria ser considerado na medição de similaridade de duas trajetórias. o SMSM provê um \emph{framework} para medição de trajetórias semânticas na medida em que cada aspecto da trajetória pode ser individualmente ajustado, como por exemplo, o grau de importância entre os \emph{stops} e os \emph{moves}, o grau de importância de cada atributo que compoe os \emph{stops} e os moves e também os limiares (\emph{thresholds}) utilizados em cada atributo para definir se houve casamento (\emph{matching}) ou não entre os pontos.
        \newline
        A medida foi avaliada em conjuntos de dados reais, já utilizadas na literatura como \cite{epfl-mobility-20090224} e também \cite{zheng2009mining}, assim como um conjunto de dados sintéticos, produzidos utilizando a ferramenta \cite{Pelekis-Hermoupolis}. 
        Inicialmente as bases de dados de trajetórias brutas foram enriquecidas com informações sobre os \emph{stops} e os \emph{moves} que ocorreram durante cada trajetória. Este enriquemento adiciona às trajetórias informações sobre os locais (POIs) em que cada \emph{stop} ocorreu, assim como informações sobre a atividade e o meio de transporte que foram utilizados durante cada \emph{move}.
        De posse das trajetórias semanticamente enriquecidas, foi possível avaliar e comparar a medida proposta. Para isto foi utilizada a abordagem de precisão em diferentes níveis de cobertura em tarefas de recuperação da informação \cite{BaezaYatesRibeiroNeto2011}. Tambem foi avaliado o impacto dos parâmetros de grau de importância e limiares para a medida proposta.
        \newline
        \newline
        
        \textbf{RESULTADOS E DISCUSSÃO}
        \newline
        \newline
        Os resultados obtidos com os experimentos realizados com os conjuntos de dados e demais medidas de similaridade supracitadas evidenciam que a medida SMSM mostrou-se a mais robusta para avaliar a similaridade de trajetórias semânticas onde tanto as informações sobre os \emph{stops} quanto os \emph{moves} são relevantes para a similaridade das trajetórias. A medida também mostrou-se flexível para suportar múltiplas dimensões de dados tanto nos \emph{stops} quanto nos \emph{moves} e flexível também para a definição de limiares (\emph{matching}) para cada dimensão. Estes resultados foram obtidos pela capacidade de a medida mensurar a similaridade de duas trajetórias utilizando também a informação do \emph{move} que ocorre entre dois \emph{stops} sequenciais.
        
        \newline
        \newline
        \textbf{CONSIDERAÇÕES FINAIS}
        \newline
        \newline
        As principais contribuições desta dissertação foram: (i) uma medida de similaridade para sequências multidimensionais para elementos com dimensões heterogêneas, como o são os \emph{stops} e \emph{moves}; (ii) a medida de similaridade suporta tanto os \emph{stops} quantos os \emph{moves}, considerando suas dimensões espaciais, temporais e semânticas, permitindo o uso de diferentes funções de distância para cada dimensão; e (iii) a medida de similaridade é flexível suficiente para considerar a ordem dos \emph{stops} e suporte a diferentes pesos para os \emph{stops}, os \emph{moves} e as dimensões constituintes de cada elemento, possibilitando se dar maior ou menor importância para cada elemento. A medida proposta nesta dissertação foi publicada no periódico International Journal of Geographical Information Science.
        \imprimirpalavraschave{Palavras-chaves}{\begin{inparaitem}[]\palavraschaveportugues\end{inparaitem}}

    \end{resumo}
    \end{otherlanguage*}
}
{
    % Changing babel package inside a single chapter
    % https://tex.stackexchange.com/questions/20987/changing-babel-package-inside-a-single-chapter
    %
    % Multiple-language document - babel - selectlanguage vs begin/end{otherlanguage}
    % https://tex.stackexchange.com/questions/36526/multiple-language-document-babel-selectlanguage-vs-begin-endotherlanguage
    \addtotextpreliminarycontent{English's Abstract}
    \begin{otherlanguage*}{english}
    \begin{resumo}[Abstract]

        For many years trajectory similarity research has focused on raw trajectories, considering only space and time information. With the trajectory semantic enrichment emerged the need for similarity measures that support space, time, and semantics. Although some trajectory similarity measures deal with all these dimensions, they consider only the stops, ignoring the moves. We claim that, for some applications, the movement between stops is as important as the stops, and it must be considered in the similarity analysis.
        In this thesis we propose SMSM, a novel similarity measure for semantic trajectories that considers both stops and moves.
        We evaluate SMSM with three trajectory datasets: (i) a synthetic trajectory dataset generated with the Hermoupolis semantic trajectory generator, (ii) a real trajectory dataset from the CRAWDAD project, and (iii) the Geolife dataset. The results show that SMSM overcomes state-of-the-art measures developed for both raw and semantic trajectories.

        \imprimirpalavraschave{Keywords}{\begin{inparaitem}[]\palavraschaveingles\end{inparaitem}}

    \end{resumo}
    \end{otherlanguage*}
}